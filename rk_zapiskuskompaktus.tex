\documentclass[a4paper,8pt]{extarticle}
\usepackage[utf8]{inputenc}


\usepackage{fancyhdr}

\usepackage[pdftex]{graphicx} % Required for including pictures
\usepackage[pdftex,linkcolor=black,pdfborder={0 0 0}]{hyperref} % Format links for pdf
\usepackage{calc} % To reset the counter in the document after title page
\usepackage{enumitem} % Includes lists

\usepackage{textcomp}
\usepackage{eurosym}

\usepackage{ dsfont } % font za množice
% tabele
\usepackage{array}
\usepackage{wrapfig}

\usepackage{tikz,forest}
\usetikzlibrary{arrows.meta}

\frenchspacing % No double spacing between sentences
\setlength{\parindent}{0pt}
\setlength{\parskip}{1.3em}

\usepackage{mathtools}
\usepackage{blkarray, bigstrut} %


\usepackage{amssymb,amsmath,amsthm,amsfonts}
\usepackage{multicol,multirow}
\usepackage{calc}
\usepackage{ifthen}
\usepackage{tabularx}
\usepackage[landscape]{geometry}
%\usepackage[colorlinks=true,citecolor=blue,linkcolor=blue]{hyperref}
%\usepackage{accents}


\newcommand{\vect}[1]{\accentset{\rightharpoonup}{#1}}

\ifthenelse{\lengthtest { \paperwidth = 11in}}
    { \geometry{top=.5in,left=.5in,right=.5in,bottom=.5in} }
	{\ifthenelse{ \lengthtest{ \paperwidth = 297mm}}
		{\geometry{top=1cm,left=1cm,right=1cm,bottom=1cm} }
		{\geometry{top=1cm,left=1cm,right=1cm,bottom=1cm} }
	}
\pagestyle{empty}
\makeatletter
\renewcommand{\section}{\@startsection{section}{1}{0mm}%
                                {-1ex plus -.5ex minus -.2ex}%
                                {0.5ex plus .2ex}%x
                                {\normalfont\large\bfseries}}
\renewcommand{\subsection}{\@startsection{subsection}{2}{0mm}%
                                {-1explus -.5ex minus -.2ex}%
                                {0.5ex plus .2ex}%
                                {\normalfont\normalsize\bfseries}}
\renewcommand{\subsubsection}{\@startsection{subsubsection}{3}{0mm}%
                                {-1ex plus -.5ex minus -.2ex}%
                                {1ex plus .2ex}%
                                {\normalfont\small\bfseries}}
\makeatother
\setcounter{secnumdepth}{0}
%\setlength{\parindent}{0pt}
%\setlength{\parskip}{0pt plus 0.5ex}
% -----------------------------------------------------------------------

\usepackage{array}
\newcolumntype{L}[1]{>{\raggedright\let\newline\\\arraybackslash\hspace{0pt}}m{#1}}
\newcolumntype{C}[1]{>{\centering\let\newline\\\arraybackslash\hspace{0pt}}m{#1}}
\newcolumntype{R}[1]{>{\raggedleft\let\newline\\\arraybackslash\hspace{0pt}}m{#1}}

\title{Računalniške komunikacije}

\begin{document}

\raggedright
\footnotesize

\begin{multicols}{4}
\setlength{\premulticols}{1pt}
\setlength{\postmulticols}{1pt}
\setlength{\multicolsep}{1pt}
\setlength{\columnsep}{2pt}

\section{ISO-OSI medel}
\renewcommand{\arraystretch}{1.5}
\begin{tabular}{|C{3.5em}|R{6.5em}|L{6.5em}|C{2.5em}|}
	\hline
	\textit{Datagr.} & \textbf{ISO-OSI} & \textbf{TCP-IP} & \textit{prot.} \\ \hline
	podatki & 	 aplikacijska & \multirow{3}{5em}{aplikacijska} & \multirow{3}{5em}{HTTP, DNS, POP3, SSH} \\ \cline{0-1}
	podatki & 	 predstavitvena & &\\ \cline{0-1}
	podatki & 	 sejna & & \\ \hline
	segmenti & 	 prenosna & transportna & TCP, UDP\\ \hline
	paketi & 	 omrežna & internetna & IP, ICMP\\ \hline
	okvirji & 	 povezavna & \multirow{2}{5em}{povezovalna} & \multirow{2}{5em}{MAC,\\ PPP}\\ \cline{0-1}
	biti & 	 	 fizična &  &\\ \hline
\end{tabular}

\section{Fizična plast}
Naloge:
\begin{itemize}
	\item \textbf{kodiranje bitov} z neko fizikalno veličino (\emph{signalom}) za prenos po mediju (\emph{baker, optika, radijski, IR, ...})
	\item \textbf{prenos posameznih bitov} v analogni ali digitalni obliki
	\item \textbf{prenos celotnega signala} (zaporedja bitov po mediju)
	\item \textbf{pretvorba signala} v obliko, ki je primerna za prenos po mediju
\end{itemize}
\subsection{Modulacija}
\begin{itemize}
\item \emph{amplitudna} (1: velika amplituda, 0: majhna ampituda)
\item \emph{frekvenčna} (1: visoka frekvenca, 0: nizka frekvenca)
\item \emph{fazna} (1: zakodiramo z fazo 0°, 0: z fazo 180°)
\item \emph{kvadratna}
	\begin{itemize}
	\item kombinacija amplitudne in fazne modulacije
	\item 4 fazni koti 0°, 90°, 180° 270°
	\item 2 nivoja amplitude
	\item 8 kombinacij faze in amplitude $\rightarrow$ 3 bite
	\item lahko imamo tudi več faznih kotov in amplitud in naenkrat
	zakodiramo do 6 bitov
	\end{itemize}
\end{itemize}

\section{Povezavna plast}
Lahko (ne pa njuno) opravlja naloge: okvirajanje datagramov, zaznavanje in odpravljanje napak, dostop do medija, zagotavljanje zanesljive dostave, kontrola pretoka.
\subsection{Okvirjanje datagramov}
Okvir je \emph{enota} na povezavni plasti. Določa začetek in konec prenesenih podatkov.

\subsection{Zaznavanje in odpravljanje napak}
dodamo bite za preverjanje pravilnosti (EDC - Error Detection Code)
\begin{itemize}
	\item \textbf{soda/liha pariteta} vseh 1 skupaj s pariteto je sodo/liho \emph{(omogoča le zaznavanje lihega števila napak)}
	\item \textbf{dvo dimenzionalna pariteta} dodamo praitetni bit za vsako
	vrstico in vsak stolpec + kumulativni/meta paritetni bit je
	pariteta paritetnih bitiv \emph{(omogoča zaznavanje in odpravljanje enojnih napak in zaznavanje dvojnih napak)}
	\item \textbf{Hammingova koda}\\
	Paritetni biti $p$ imajo indekse $2^i$ in jih postavimo na mesta, ki jih določajo indeksi. Vmes vstavimo bite podatkov $d$.\\
	Vsak paritetni bit $p_i$ pokriva $i$ bitov od svoje pozicije naprej, nato preskoči $i$ bitiov in tako naprej. \\
	Število 1, ki jih paritetni bit pokriva mora biti sodo.
	\begin{tabular}{c | *{7}{c}}
			& $1.$ & $2.$ & $3.$ & $4.$ & $5.$ & $6.$ & $7.$ \\ \hline
			& $p_1$ & $p_2$ & $d_1$ & $p_4$ & $d_2$ & $d_3$ & $d_4$ \\ \hline
		$p_1$	&\checkmark& &\checkmark& &\checkmark& &\checkmark \\
		$p_2$	& &\checkmark&\checkmark& & &\checkmark&\checkmark \\
		$p_4$ 	& & & &\checkmark&\checkmark&\checkmark&\checkmark
	\end{tabular}

	Ko prejmomo sporočio, preverimo paritetne bite in definiramo:
	\begin{align*}	
		p_i' = \begin{cases}
			1 & \text{paritetni bit $p_i$ napačen}\\
			0 & \text{sicer}
		\end{cases}
	\end{align*}
	Nato izračunamo sindrom:
	\begin{align*}
		 [ \dots p'_4\ p'_2\ p'_1 ]_{(2)} = \text{\# bita, ki ga je treba obrniti}
	\end{align*} 
	\item \textbf{Cyclic Redundancy Check - CRC} \emph{z $r$ paritetnimi biti lahko popravi do $r+1$ napak}
	\item \textbf{Internet checksum} \emph{uporablja se na transportni plasti}
\end{itemize}

\subsection{Dostop do skupinskega medija}
Imamo dve vrsti povezav: \textbf{dvotočkovna} (le en pošiljatel in prejemnik - PPP, DHLC) in \textbf{oddajna} (več vozlišč komunicira naenkrat - Ethernet, Wireless LAN).

Kanal lahko delimo na več načinov:
\subsubsection{Delitev kanala}
Time Division Multiple Access TDMA, Frequency Devision Multiple Access FDMA \emph{Ni kolizij}
\subsubsection{Sočasni/naključni dostop (kolizijski)}
\begin{itemize}
	\item \textbf{ALOHA} Če pride do kolizije, se paket po naključnem času ponovno pošlje.
	\item \textbf{razsekana ALOHA} Čas je razsekan na časovne intervale. V vsakem intervalu se lahko pošlje en paket. Če pride do kolizije, se v nasldnjem intervalu pošlje ponovno z verjetnostjo $p$.
	\item \textbf{Carier Sense Multiple Access - CSMA} Poslušaj ali že kdo oddaja preden začeneš oddajati sam.
	\item \textbf{Carier Sense Multiple Accerss Collision Detection -  CSMA/CD} Če zaznaš kolizijo nehaj oddajati in pošlji motilni siganal.
\end{itemize}

\subsubsection{Izmenični dostop}
\begin{itemize}
	\item \textbf{Rezervacija s centralnim vozliščim (pooling)}: centralno vozlišče zaporedno dodeluje medij posameznim vozliščem (\emph{zakasnitev, ena točka odpovedi})
	\item \textbf{Rezervacija z žetonom, ki kroži (token)} (\emph{zakasnitev, ena točka odpovedi - žeton})
\end{itemize}

\subsubsection{Dostop do brezžičnega medija}
Protokol 802.11 (WiFi) za dostop do medija uporablja \textbf{Carier Sense Multiple Accerss Collision Avoidance - CSMA-CA}.

Uporablja signala:
\begin{itemize}
	\item \textbf{Request To Send - RTS}
	\item \textbf{Clear To Send - CTS}
\end{itemize}
Terminal, ki želi pošiljati najprej pošlje RTS in čaka na CTS.
Ko terminal prejme CTS, ne bo oddajal toliko časa kot traja, da se prenese en paket.

\subsection{Protokoli na povezavni plasti}
\subsubsection{Ethernet}
Okvir ethernet:

\setlength\tabcolsep{1.5pt}
\begin{tabular}{*{6}{| C{3.5em}} |}
	\hline
	\tiny{\textit{preambula}} & \tiny{\textit{ponorni nalosv}} & \tiny{\textit{izvorni naslov}} & \tiny{\textit{tip}} & \tiny{\textit{podatki}} & \tiny{\textit{CRC}} \\ 
	\tiny{\texttt{8B}} & \tiny{\texttt{6B}} & \tiny{\texttt{6B}} & \tiny{\texttt{2B}} & \tiny{\texttt{1500B}} & \tiny{\texttt{4B}} \\ \hline 
\end{tabular}

Ethernet ponuja:
\begin{itemize}
	\item nepovezavna storitev
	\item nezanesljiva storitev (\emph{kontrola pravilnosti se izvaja s CRC
	a brez popravljanja, potrjevanje in ponovno pošiljanje se ne
	uporabljata})
	\item uporablja CSMA/CD (\emph{pri vsaki koliziji se čakanje eksponentno poveča $0-1,0-3,0-7,0-15,\dots$})
\end{itemize}

\subsubsection{Point to Point Protocol - PPP}
Povezavni protokol; imamo le enega pošiljatelja in prejemnika.

Okvir PPP:

\setlength\tabcolsep{1.5pt}
\begin{tabular}{*{7}{| C{3.2em}} |}
	\hline
	\tiny{\textit{zastavica za začetek}} & \tiny{\textit{address}} & \tiny{\textit{control}} & \tiny{\textit{protokol}} & \tiny{\textit{podatki}} & \tiny{\textit{CRC}} & \tiny{\textit{zastavica za konec}} \\ 
	\tiny{\texttt{1B}} & \tiny{\texttt{1B}} & \tiny{\texttt{1B}} & \tiny{\texttt{1B/2B}} & \tiny{\texttt{$\leq$1500B}} & \tiny{\texttt{2B/4B}} & \tiny{\texttt{1B}} \\ \hline 
\end{tabular}

Poja ``address'' in ``control'' se ne uporabljata.

Pred niz \texttt{01111110} (zastavica) urivamo \emph{ubežno kodo} \texttt{01111101}.

\subsection{Naslavljanje naprav na povezavni plasti}
Vsaka naprava ima fizični naslov (Media Access Control - MAC) iz 48 bitov.

Naslov \texttt{FF-FF-FF-FF-FF-FF} je \textbf{brodcast} (prejemniki so vsi). 

\subsubsection{Address Resolution Protocol - ARP}
Vsaka naprava ima ARP tabelo [IP $\vert$ MAC $\vert$ TTL]

Če naprava želi ugotoviti MAC druge naprave, pošlje \textbf{ARP poizvedbo} (\emph{Moj max je y, kdo ima IP x?}) na brodcast naslov. Naparava, ki ima ta IP odgovori z \textbf{ARP odgovorom} (\emph{Jaz imam IP x in oj MAC je z.}). Nato si obe napravi shranita MAC naslove v ARP tabelo.

\subsection{Stikalo}
Stikalo uporablja \textbf{stikalno tabelo} [MAC $\vert$ vrata $\vert$ TTL], da se odloči, na katera vrata poslati okvir. Ko prejme kak okvir, si zapomni lokacijo pošiljatelja in okvir:
\begin{itemize}
	\item \textbf{poplavi} na vsa vrata razen na tista iz katerih je prišel okvir (\emph{če ne ve, na katerih vratih je prejemnik})
	\item \textbf{posreduje} na izbrana vrata (\emph{če ve, na katerih vratih je prejemnik})
	\item \textbf{filtriria} / zavrže okvir (\emph{če je namenjen istim vratom iz katerih je prišel})
\end{itemize}

\section{Omrežna plast}
\emph{Lahko} omogoča:
\begin{enumerate}
	\item zagotovljena dostava paketov
	\item dostava v zagotovljenem času
	\item dostava v pravem vrstnem redu
	\item zagotovljena spodnja meja pasovne širine
	\item največja dovoljena varianca zakasnitve (jitter)
	\item varno komunikacijo (zaupnost integriteto podatkov, avtentikacijo)
\end{enumerate}

\subsection{Storitve interneta}
\setlength\tabcolsep{1.5pt}
\begin{tabular}{c C{4em} *{5}{|C{3.2em}}|}
	& & \multicolumn{5}{c|}{\emph{zagotovljene storitve}} \\ \hline
	\tiny{Omrežje} & \tiny{Model} & \tiny{pas. širina} & \tiny{brez izgub} & \tiny{vr. red} & \tiny{čas} & \tiny{obv. o zamaš.} \\ \hline
	\emph{Internet} & best effort & $\times$ & $\times$ & $\times$ & $\times$ & $\times_{\text{izguba}}$ \\
	\emph{ATM} & CBR \tiny{constant bit rate} & const & \checkmark & \checkmark & \checkmark & \tiny{ni zamaš.} \\
	\emph{ATM} & ABR \tiny{available bit rate} & min & $\times$ & \checkmark & $\times$ & \checkmark \\
\end{tabular}
\subsection{Usmerjevalnik}
Naprava, ki deluje na omrežni plasti in skrbi za transport datagrama po \emph{jedru omrežja}. Povezuje ralične medije in portokole.

Funkcije usmerjevalnika:
\begin{itemize}
	\item \textbf{Posredovanje paketov}/forwarding: prenos na pravi izhodni vmesnik glede na destinacijo paketa in \textbf{posredovalno tabelo}.
	\item \textbf{Usmerjanje}/routing: določanje poti paketov od izvora do cilja. Usmerjevalniki s poočjo usmerjevalnih protokolov najdejo najcenejšo pot.
\end{itemize}

\subsubsection{Povezavna omrežja / navidiezni vodi}
Faze pri vzpostavitvi navideznega voda: \emph{vzpostavitev}, \emph{tok podatkov}, \emph{rušenje}

Usmerjevalniki usmerjajo pakete glede na \emph{št. voda} v paketu in posredovalno tabelo [ vhodni vmesnik $\vert$ vhodna št. voda $\vert$ izhodni vmesnik $\vert$ izhodna št. voda].

\subsubsection{Nepovezavna / datagramska omrežja}
Ni faze vspostavljanja povezave. Paket lahko do cilja pride po različnih poteh.

Usmerjevalniki usmerjajo glede na ciljni (IP) naslov in usmerjevalno tabelo [predpona naslova $\vert$ vmesnik]. 

\subsection{Internet Protocol - IP}
\subsection{Paket IPv4}
Dolžina glave brez ``možnosti'' je 20B.

\setlength\tabcolsep{1.5pt}
\begin{tabular}{*{8}{|C{20pt}}|}
	\hline
	\texttt{4b} & \texttt{4b} & \texttt{4b} & \texttt{4b} & \texttt{4b} & \texttt{4b} & \texttt{4b} & \texttt{4b} \\ \hline
	\tiny{\textit{Verzija}} & \tiny{\textit{Dolžina glave}} & \multicolumn{2}{c|}{\tiny{\textit{Tip storitve}}} & \multicolumn{4}{c|}{\tiny{\textit{Skupna dolžina}}} \\ \hline
	\multicolumn{4}{|c|}{\tiny{\textit{Identifikacija}}} & \tiny{\textit{Zastav.}} & \multicolumn{3}{c|}{\tiny{\textit{Zamik fragmenta}}} \\ \hline
	\multicolumn{2}{|c|}{\tiny{\textit{TTL}}} & \multicolumn{2}{c|}{\tiny{\textit{Protokol}}} & \multicolumn{4}{c|}{\tiny{\textit{Kontrolna vsota}}} \\ \hline
	\multicolumn{8}{|c|}{\textit{Izvorni IP naslov} \texttt{32b}} \\ \hline
	\multicolumn{8}{|c|}{\textit{Ponorni IP naslov} \texttt{32b}} \\ \hline
	\multicolumn{6}{|c|}{\tiny{\textit{Možnosti}}} &  \multicolumn{2}{c|}{\tiny{\textit{Zpolnjevanje}}} \\ \hline
	\multicolumn{8}{|c|}{\tiny{\textit{Podatki}}} \\ \hline
\end{tabular}

\subsubsection{Fragmentacija}
IP datagram se mora enkapsulirati v omejene okvirje (MTU) povezavne plasti. Zato se IP paket razbije na več manjših.
Da fragmente lahko sestavimo nazaj se uporabljajo polja v glavi:

\begin{itemize}
	\item \textbf{ID}: pri vseh fragmentih enak
	\item \textbf{Zastavica MF}: 0, če je to zadnji fragment; 1, sicer
	\item \textbf{Offset}: enota za odmik je 8 Byte
\end{itemize}

\subsubsection{Podomrežja}
\begin{tabular}{|c|l|l|}
	\hline
	\textbf{Razred} & \textbf{Začetek naslova} & \textbf{Maska} \\ \hline
	A &	\texttt{0 xxxxxxx} 	& \texttt{255.0.0.0} \\ \hline
	B &	\texttt{10 xxxxxx} 	& \texttt{255.255.0.0} \\ \hline
	C &	\texttt{110 xxxxx} 	& \texttt{255.255.255.0} \\ \hline
	D &	\texttt{1110 xxxx} 	& \texttt{-} \\ \hline
	E &	\texttt{11110 xxx} 	& \texttt{-} \\ \hline
\end{tabular}

\textbf{Posebni naslovi}
\begin{itemize}
	\item Privatni (24 bitni) \texttt{ 10.0.0.0/8}
	\item Privatni (20 bitni)\texttt{ 172.16.0.0/12}
	\item Privatni (16 bitni)\texttt{ 192.168.0.0/16}
	\item Povratna zanka (localhost) \texttt{ 127.0.0.0/8}
	\item Link-local \texttt{ 169.254.0.0/16}
\end{itemize}

\textbf{Posebni naslovi v lokalnem omrežju}
\begin{itemize}
	\item Naslov omrežja: \emph{1. naslov v omrežju (same ničle)}
	\item Hosts: \emph{vsi naslovi vmes}
	\item Brodcast: \emph{zadnji naslov v omrežju (same enke)}
\end{itemize}

\subsection{Dinaminčno dodeljevanje naslovov - DHCP}
Ko se naprava priključi v omrežje, pošlje \textbf{DHCP discover}, DHCP strežnik(i) odgovori(jo) z \textbf{DHCP offer}, naprava izbere IP iz ponudbe in pošlje \textbf{DHCP request}, končno strežnik potrdi z \textbf{DHCP ACK}.

\subsection{Internet Control Message Protocol - ICMP}
Sporočilo je enkapsulirano v paketu IP. Uporablja se za sporočanje napake, nedosegljivosti, protokola, vrat, ...

\textbf{Traceroute} pošlje vrsto paketov z TTL=1,2,3,... vsakič ko paketu poteče TTL router pošlje nazaj ICMP sporočilo.

\subsection{Network Address Translation - NAT}
Lokalne naslove slika v globalne in obratno.

To doseže z NAT preslikovalno tabelo [globalni naslov, port $\vert$ zasebni naslov, port].

Vsak zahtevek ima izvorni in ponorni port/vrata.

\subsection{IPv6}
Naslovi so razdeljeni (:) na 8 blokov po 16 bitov. Zaporedje blokov samih ničel lahko krajšamo z ``::'', 

Dolžina glave je vedno 40B.
\begin{tabular}{*{8}{|C{20pt}}|}
	\hline
	\texttt{4b} & \texttt{4b} & \texttt{4b} & \texttt{4b} & \texttt{4b} & \texttt{4b} & \texttt{4b} & \texttt{4b} \\ \hline
	\tiny{\textit{Verzija}} & \multicolumn{2}{c|}{\tiny{\textit{Traffic class}}} & \multicolumn{5}{c|}{\tiny{\textit{Flow label}}} \\ \hline
	\multicolumn{4}{|c|}{\tiny{\textit{Payload len}}} & \multicolumn{2}{c|}{\tiny{\textit{Next header}}} & \multicolumn{2}{c|}{\tiny{\textit{Hop limit}}} \\ \hline
	\multicolumn{8}{|c|}{\textit{Izvorni IP naslov} \texttt{128b}} \\ \hline
	\multicolumn{8}{|c|}{\textit{Ponorni IP naslov} \texttt{128b}} \\ \hline
	\multicolumn{8}{|c|}{\tiny{\textit{Podatki}}} \\ \hline
\end{tabular}
Novo polje ``flow label'' (vrsta toka) za zagotavljanje kakovosti storitev za posebne tokove podatkov.

\subsubsection{Prehodni mehanizmi}
\begin{itemize}
	\item \textbf{Dvojni slkad}: Vse naprave imajo IPv4 in IPv6. Če je na poti kakšno IPv4 vozlišče, se bo proromet vmes pretvarjal v IPv4.
	\item \textbf{Tuneliranje}: Če je na poti kakpno IPv4 vozlišče se IPv6 paket enkapsulira v IPv4.
\end{itemize}

\subsection{Usmerjanje}
Usmerjevalni algoritmi so lahko: 
\begin{itemize}
	\item \textbf{centralizirani} (\emph{imajo podatke o celem omrežju}) ali \textbf{decentralizirani} (\emph{imajo podatke le o sosednjih vozliščih}) \\
	\item \textbf{prilagodljivi} (\emph{lahko prilagajajo cene povezav glede na zasičenost}) ali \textbf{naprilagodljivi}
\end{itemize}

\subsubsection{Decentralizirano usmerjanje}
Vsako vozlišče pozna ceno povezave do svojih sosedov. Hrani \textbf{vektor razdalj}, \textbf{vektorje razdalj svojih sosedov} in \textbf{posredovalno tabelo} [cilj $\vert$ cena $\vert$ via sosed].

Ko vozlišče prejme vkektor razdalj svojega soseda, izračuna nov vektor razdalj \emph{če lahko do nekega vozlišča pride ceneje prek nekega soseda, posodobi posredovalno tabelo}.

Usmerjevalniki so povezani v \textbf{avtonomne sisteme - AS} v katerih uporabljajo isti \textbf{INTRA-AS} usmerjevalni protokol.
\begin{itemize}
	\item \textbf{Routing Information Protocol - RIP} usmerjanje z vekotrjem razdalj (na 30s, 180s+ $\rightarrow$ prekinjena povezava), cena je število skokov, max cena je 15M
	\item \textbf{Open Shortest Path First - OSPF} obvestila se s poplavljanjem posredujejo po celem sistemu, vsak potem preračuna najkrajše poti
	\item \textbf{Interior Gateway Routing Protocol - IGRP} Cisco-va izboljšava RIP; cena je odvisna od pasovne širine, zakasnitve, obremenitve, MTU in zanesljivosti.
\end{itemize}

Za povezovanje AS mad seboj se uporablja \textbf{INTER-AS} usmerjevalni protokol, ki pa morabiti enak v celem omrežju.
\begin{itemize}
	\item \textbf{Border Gateway Protocol - BGP} omogoča, da omrežja oglašujejo svojo prisotnost drugim omrežjem
\end{itemize}


\section*{Transportna plast}
Naloge transportne plasti so:
\begin{itemize}
	\item povezovanje dveh oddaljenih \emph{procesov}
	\item multipleksiranje/demultipleksiranje komunikacije med procesi
	\item zanesljiv prenos podatkov
	\item kontrola pretoka in zasičenja
\end{itemize}

\subsection{Vrata - port}
FTP-data: 20, FTP-cmd: 21, SSH: 22, Telnet: 23, SMTP: 25, DNS: 53, HTTP: 80, POP3: 110, IMAP: 143, IRC: 194, HTTPS: 443

\subsection{User Datagram Protocol: UDP}
Nepovezavna storitev, nudi le best-effort: ni zagotavljanja vrstnega reda, ni nadzora zamašitev, ...

Dolžina glave je le 8B.

\begin{tabular}{*{2}{|C{80pt}}|}
	\hline
	\texttt{16b} & \texttt{16b} \\ \hline
	\tiny{\textit{Izvorna vrata}} & \tiny{\textit{Ponorna vrata}} \\ \hline
	\tiny{\textit{Dolžina (skupaj z glavo) v B}} & \tiny{\textit{Internetna kontrolna vsota}} \\ \hline
\end{tabular}

\subsubsection{Internetna kontrolna vsota}
\emph{Pošiljatelj:} pošiljatelj sešteje 16 bitne besede in shrani eniški komplement = kontrolna vsota \\
\emph{Prejemnik:} sešteje 16 bitne besede skupaj s kontrolno vsoto $\to$ dobiti mora same enice

\subsection{Potrjevanje}
\textbf{sprotno potrjevanje}: po vsakem pošiljanju počakaj na potrditev

\textbf{tekoče pošiljanje:} večji razpon za številčenje paketov, shranjevanje paketov na obeh straneh
\begin{itemize}
	\item ponavljanje N potrjenih (\emph{go-back-N})
	\begin{itemize}
		\item pošiljatelj hrani okno največ dovoljenih nepotrjenih paketov
		\item ACK($n$) potrdi vse pakete do vključno $n$
		\item časovna kontrola za najstarejši paket, ko poteče pošlji vse nepotrjene pakete na novo
	\end{itemize}
	\item ponavljanje izbranih (\emph{selective repeat})
	\begin{itemize}
		\item prejemnik shranjuje sporočila in jih sortira pred dostavo
		\item pošiljatel ponovno pošlje le tiste pakete za katere ni prejel potrdila
		\item vsak paket ima svojo časovno kontrolo
	\end{itemize}
\end{itemize}

\textbf{neposredno}: uporaba ACK in NCK
\textbf{posredno}: samo ACK, namesto NCK se ponovi ACK zadnjega segmenta



\subsection{Transfer Control Protocol: TCP}
Povezavna storitev, full duplex, zanesljiv, kontrola pretoka, kontrola zasičenja, \emph{tekoče pošiljanje in neposredno potrjevanje}

\begin{tabular}{*{32}{|C{1pt}}|}
	\hline
	\multicolumn{16}{|c|}{\tiny{\texttt{16b}}} & \multicolumn{16}{c|}{\tiny{\texttt{16b}}} \\ \hline
	\multicolumn{16}{|c|}{\tiny{\textit{Izvorna vrata}}} & \multicolumn{16}{c|}{\tiny{\textit{Ponorna vrata}}} \\ \hline
	\multicolumn{16}{|c|}{\tiny{\textit{Dolžina (skupaj z glavo) v B}}} & \multicolumn{16}{c|}{\tiny{\textit{Internetna kontrolna vsota}}} \\ \hline
	\multicolumn{32}{|c|}{\tiny{\textit{zaporedna št. (B)}}} \\ \hline
	\multicolumn{32}{|c|}{\tiny{\textit{št. potrditve (B)}}} \\ \hline
	%\tiny{\textit{Kontrolna vsota}} & \tiny{\textit{sprejemno okno}} \\ \hline
	\multicolumn{4}{|c|}{\tiny{\textit{Dolzina glave}}} & \multicolumn{3}{c|}{\tiny{\textit{0}}} & \multicolumn{9}{c|}{\tiny{\textit{zastavice}}} & \multicolumn{16}{c|}{\tiny{\textit{sprejemno okno}}} \\ \hline
	\multicolumn{16}{|c|}{\tiny{\textit{Internetna kontrolna vsota}}} & \multicolumn{16}{c|}{\tiny{\textit{karalec urg. data}}} \\ \hline
	\multicolumn{32}{|c|}{\tiny{\textit{opcije (spremenljiva dolžina)}}} \\ \hline
\end{tabular}

Dolžina glave (4b) je podana v številu 32-bitnih besede

Sprejemno okno (16b) št. bajtov, ki jih sprejemnik lahko sprejme

Zastavice (9b):
\begin{itemize}
	\item URG*: urgentni podatki
	\item ACK: potrditev povezave
	\item PSH: takoj predaj aplikaciji
	\item RST, SYN, FIN: vzpostavljanje in rušenje povezave
\end{itemize}

\subsubsection{Vzpostavitev povezave - trojno rokovanje}
\begin{enumerate}
	\item Odjemalec pošlje segment z zastavico \texttt{SYN} (sporoči začetno številko segmenta, ni podatkov)
	\item Strežnik vrne segment \texttt{SYN ACK} (rezervira medpomnilnik, odgovori z začetno številko svojega segmenta)
	\item Odjemalec vrne \texttt{ACK}, lahko že s podatki ("štuporama")
\end{enumerate}

\subsubsection{Rušenje povezave}
\begin{enumerate}
	\item odjemalec pošlje segment  \texttt{TCP FIN} strežniku
	\item strežnik potrdi z  \texttt{ACK}, zapre povezavo, pošlje  \texttt{FIN}
	\item odjemalec prejme strežnikov  \texttt{FIN}, potrdi ga z  \texttt{ACK} (počaka časovni interval, da po potrebi ponovno pošlje  \texttt{ACK}, če se ta izgubi)
	\item strežnik sprejme  \texttt{ACK}, končano
\end{enumerate}

\subsubsection{Nastavitev časovne kontrole}
Na potrditev moramo čakati vsaj \texttt{RTT} (Round Trip Time).

\begin{gather*}
	\texttt{OcenRTT}[i] = (1-\alpha) \texttt{OcenRTT}[i-1] + \alpha  \texttt{IzmerRTT}[i] \\
	\texttt{DevRTT}[i] = (1-\beta) \texttt{DevRTT}[i-1] + \beta  \big| \texttt{IzmerRTT}[i] - \texttt{OcenRTT}[i] \big| \\
	\texttt{ČakalniInterval}[i] = \texttt{OcenRTT}[i] + 4  \texttt{DevRTT}[i]
\end{gather*}

\subsubsection{Način potrjevanja}
TCP uporablja podobno strategijo kot \emph{ponavljanje N nepotrjenih}. Pošiljatelj ima časovno kontrolo le za najstarejši nepotrjen segment, vendar ob poteku časovne kontrole ne pošlje vseh segmentov v oknu, temveč le najstarejši nepotrjen segment.

RFC2018 vpeljuje potrjevanje le izbranih paketov.

\begin{tabularx}{\columnwidth}{|X|X|}
	\hline
	\emph{Dogodek pri prejemniku} & \emph{Odziv prejemnika} \\ \hline
	Sprejem segmenta s pričakovano številko, vsi prejšnji že potrjeni. &
	Počakaj na naslednji segment max 500 ms. Če ta pride v tem intervalu, izvedi zakasnjeno potrditev obeh (\emph{delayed ACK}). Če ne pride v tem intervalu, potrdi samo prejetega. \\ \hline
	Isto kot zgoraj, a potrditev za prejšnji segment še ni bila poslana. &
	Takoj pošlji \emph{kumulativno potrditev} za oba segmenta brez izvajanja zakasnjene potrditve. \\ \hline
	Sprejem segmenta s previsoko številko (\emph{zaznamo vrzel}) &
	Takoj potrdi zadnji še sprejeti segment (\emph{pošlji duplikat ACK}). \\ \hline
	Sprejem segmenta z najnižjo	številko iz vrzeli (\emph{polnjenje vrzeli}) &
	Takoj potrdi segment. \\ \hline
\end{tabularx}

\subsubsection{Hitro ponovno pošiljanje (fast retransmit)}
Ponovno pošiljanje se običajno izvede po preteku časovne kontrole. Če pa pošiljatel prejme 3 potrditve za že potrjen paket, takoj izvede ponovno pošiljanje.

\subsubsection{Kontrola pretoka}
Usklajevanje hitrosti pošiljanja med pošiljateljem in prejemnikom.

Prejemnik sporoča razpolžljiv prostor v pomnilniku v polju 'sprejemno okno':
\begin{align*}
	\texttt{rwnd} = \texttt{velikost} - [\texttt{lastByteRcvd} - \texttt{lastByteRead}]
\end{align*}

\subsubsection{Oznake}
\begin{align*}
	\texttt{rwnd} \quad  \dots & \quad \text{recieve window} \\
	\texttt{cwnd} \quad  \dots & \quad \text{congestion window} \\
	\texttt{MSS} \quad  \dots & \quad \text{maximum segment size} \\
	\texttt{RTT} \quad  \dots & \quad \text{round trip time} \\
	\texttt{ssthresh} \quad  \dots & \quad \text{slow start threshold} \\
\end{align*}

\subsubsection{Nadzor zasičenja}
$\min(\texttt{rwnd}, \texttt{cwnd})$ določa največ nepotrjenih B preden moramo ustaviti pošiljanje.

\textbf{TCP Tahoe:} osnovna verzija; 2 fazi: \emph{počasen začetek} in \emph{izogibanje zasičenju}; po izgubi paketa \texttt{cwnd} = 1 \texttt{MSS}.

\textbf{TCP Reno:} 3 faze: \emph{počasen začetek}, \emph{izogibanje zasičenju} in \emph{hitra obnova}; če dobiš 3 kopije ACK že potrjenih podatkov, \texttt{cwnd} = \texttt{cwnd}/2 + 3 \texttt{MSS}.

\textbf{TCP Vegas:} dodano zaznavanje sitacij, ki vodijo v zasičenje in \emph{linearno zmanjševanje hitrosti} ob zasičenju.


Na začetku: \texttt{cwnd} = 1\texttt{MSS}; faza počasnega začetka;\\ \texttt{ssthresh} = $\infty$.


\begin{itemize}
	\item \textbf{Faza počasnega začetka:}\\
	Za vsak prejet ACK: \texttt{cwnd} += $\min(\texttt{N},\texttt{MSS})$, kjer je \texttt{N} št. B, ki jih je potrdil ta ACK. \\
	\emph{Alt. rešitev}: Za vsak prejet ACK:  \texttt{cwnd} += \texttt{MSS}.

	\emph{Na ta način se skozi čas \texttt{cwnd} povečuje eksponentno.}
	\begin{itemize}
		\item Če je \texttt{cwnd} $\geq$ \texttt{ssthresh}, preidemo v fazo izogibanja zasičenju.
		\item Če prejmemo 3 ACK, že potrjenega segmenta, gremo v \emph{fazo hitre obnove} \texttt{cwnd} = \text{ssthresh} + 3\texttt{MSS}, \texttt{ssthresh} = \texttt{cwnd} / 2
	\end{itemize}
	
	\item \textbf{Faza izogibanja zasičenju:}\\
	Števemo B, ki jih potrdijo ACK. Ko to število preseže \texttt{cwnd}, resetiramo števec in \texttt{cwnd} += \text{MSS}. \\
	\emph{Alt. rešitev}: Za vsak ACK, ki potrdi nove podatke povečamo $\texttt{cwnd} += \texttt{MSS}\cdot\texttt{MSS} / \texttt{cwnd}$
	
	\emph{Na ta način se skozi čas \texttt{cwnd} povečuje linearno.}
	\begin{itemize}
		\item Če poteče časovna pontrola za najstarejši segment, gremo v \emph{fazo počasnega začetka} in  \texttt{cwnd} = \text{MSS}, \texttt{ssthresh} = \texttt{cwnd} / 2.
		\item Če prejmemo 3 ACK, že potrjenega segmenta, gremo v \emph{fazo hitre obnove} \texttt{cwnd} = \text{ssthresh} + 3\texttt{MSS}, \texttt{ssthresh} = \texttt{cwnd} / 2
	\end{itemize}
	
	\item \textbf{Faza hitre obnove:}\\
	Za vsak podvojeni ACK: \texttt{cwnd} += \texttt{MSS}.
	\begin{itemize}
		\item Če dobimo ACK, ki potrjuje nove podatke, gremo v \emph{fazo izogibanja zasičenju} in \texttt{cwnd} += \texttt{ssthresh}.
		\item Če preteče nek timeout, gremo v \emph{fazo počasnega začetka} in  \texttt{cwnd} = \text{MSS}, \texttt{ssthresh} = \texttt{cwnd} / 2.
	\end{itemize}
\end{itemize}


TCP konvergira k pravični delitvi pasovne širine med uporabniki.

\section{Aplikacijska plast}
\subsection{HTTP}

\subsubsection{Glava}
\texttt{Metoda URL Verzija}\\
\texttt{Ime polja 1: vrednost 1}\\
\texttt{...}\\
\texttt{Ime polja n: vrednost n}\\
\textit{[prazna vrstica]}\\
\texttt{TELO}

\subsubsection{Metode}
\begin{tabularx}{\columnwidth}{X X}
	\hline
	\texttt{GET} & zahteva objekta \\ \hline
	\texttt{POST} & zahteva objekta + poslane vrednosti (obrazci) \\ \hline
	\texttt{HEAD} & zahteva, na katero strežnik odgovori z odgovorom brez zahtevanega objekta (uporabno za razhroščevanje) \\ \hline
	\texttt{PUT} & (HTTP 1.1) – nalaganje na strežnik (upload) \\ \hline
	\texttt{DELETE} & (HTTP 1.1) – brisanje s strežnika \\ \hline
	\texttt{TRACE} & razhroščevanje (echo-odmev zahtevka, podobno PING) \\ \hline
	\texttt{CONNECT} & povezava preko medstrežnika \\ \hline
	\texttt{OPTIONS} & povpraševanje o možnih opcijah pri zahtevku \\ \hline
\end{tabularx}

\subsubsection{Statusi}
\begin{itemize}
	\item \textbf{1xx: informativne kode}
	\item \textbf{2xx: uspešno} \\
		200: OK
	\item \textbf{3xx: preusmeritev} \\
		301: Moved Permanently – prestavljen dokument 
	\item \textbf{4xx: napake pri odjemalcu} \\
		400: Bad Request – sintaksa \\
		404: Not Found – ni dokumenta
	\item \textbf{5xx: napake na strežniku} \\
		500: Internal Server Error \\
		505: HTTP Version Not Supported
\end{itemize}

\subsubsection{Vrste HTTP povezav}
\begin{enumerate}
	\item \textbf{nevztrajne (nonpersistent)}: za vsak prenašani objekt (stran, sliko) se vzpostavi nova TCP	povezava
	
	2 RTT/objekt (rokovanje + prenos)
	\item \textbf{vztrajne (persistent)}: strežnik uporabi isto povezavo za pošiljanje več objektov, strežnik pusti povezavo odprto
	
	1 RTT/objekt (prenos)
	\item \textbf{vztrajne s cevovodom (persistent, pipelined)}: tekoče pošiljanje več zahtev naenkrat, brezčakanja na prejem prejšnjih
\end{enumerate}

\subsubsection{Medstrežnik}
Pogojni zahtevek: \texttt{If-modified-since: <datum>}.

Strežnik pošlje nov stran ali pa \texttt{HTTP/1.1 304 Not Modified}
\end{multicols}
\end{document}