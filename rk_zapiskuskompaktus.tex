\documentclass[a4paper,8pt]{extarticle}
\usepackage[utf8]{inputenc}


\usepackage{fancyhdr}

\usepackage[pdftex]{graphicx} % Required for including pictures
\usepackage[pdftex,linkcolor=black,pdfborder={0 0 0}]{hyperref} % Format links for pdf
\usepackage{calc} % To reset the counter in the document after title page
\usepackage{enumitem} % Includes lists

\usepackage{textcomp}
\usepackage{eurosym}

\usepackage{ dsfont } % font za množice
% tabele
\usepackage{array}
\usepackage{wrapfig}

\usepackage{tikz,forest}
\usetikzlibrary{arrows.meta}

\frenchspacing % No double spacing between sentences
\setlength{\parindent}{0pt}
\setlength{\parskip}{0.7em}

\usepackage{mathtools}
\usepackage{blkarray, bigstrut} %


\usepackage{amssymb,amsmath,amsthm,amsfonts}
\usepackage{multicol,multirow}
\usepackage{calc}
\usepackage{ifthen}
\usepackage{tabularx}
\usepackage[landscape]{geometry}
%\usepackage[colorlinks=true,citecolor=blue,linkcolor=blue]{hyperref}
%\usepackage{accents}


\newcommand{\vect}[1]{\accentset{\rightharpoonup}{#1}}

\ifthenelse{\lengthtest { \paperwidth = 11in}}
    { \geometry{top=.5in,left=.5in,right=.5in,bottom=.5in} }
	{\ifthenelse{ \lengthtest{ \paperwidth = 297mm}}
		{\geometry{top=1cm,left=1cm,right=1cm,bottom=1cm} }
		{\geometry{top=1cm,left=1cm,right=1cm,bottom=1cm} }
	}
\pagestyle{empty}
\makeatletter
\renewcommand{\section}{\@startsection{section}{1}{0mm}%
                                {-1ex plus -.5ex minus -.2ex}%
                                {0.5ex plus .2ex}%x
                                {\normalfont\large\bfseries}}
\renewcommand{\subsection}{\@startsection{subsection}{2}{0mm}%
                                {-1explus -.5ex minus -.2ex}%
                                {0.5ex plus .2ex}%
                                {\normalfont\normalsize\bfseries}}
\renewcommand{\subsubsection}{\@startsection{subsubsection}{3}{0mm}%
                                {-1ex plus -.5ex minus -.2ex}%
                                {1ex plus .2ex}%
                                {\normalfont\small\bfseries}}
\makeatother
\setcounter{secnumdepth}{0}
%\setlength{\parindent}{0pt}
%\setlength{\parskip}{0pt plus 0.5ex}
% -----------------------------------------------------------------------

\usepackage{array}
\newcolumntype{L}[1]{>{\raggedright\let\newline\\\arraybackslash\hspace{0pt}}m{#1}}
\newcolumntype{C}[1]{>{\centering\let\newline\\\arraybackslash\hspace{0pt}}m{#1}}
\newcolumntype{R}[1]{>{\raggedleft\let\newline\\\arraybackslash\hspace{0pt}}m{#1}}

\title{Računalniške komunikacije}

\begin{document}

\raggedright
\footnotesize

\begin{multicols}{4}
\setlength{\premulticols}{1pt}
\setlength{\postmulticols}{1pt}
\setlength{\multicolsep}{1pt}
\setlength{\columnsep}{2pt}

\section*{Transportna plast}
Naloge transportne plasti so:
\begin{itemize}
	\item povezovanje dveh oddaljenih \emph{procesov}
	\item multipleksiranje/demultipleksiranje komunikacije med procesi
	\item zanesljiv prenos podatkov
	\item kontrola pretoka in zasičenja
\end{itemize}

\subsection{Vrata - port}
FTP-data: 20, FTP-cmd: 21, SSH: 22, Telnet: 23, SMTP: 25, DNS: 53, HTTP: 80, POP3: 110, IMAP: 143, IRC: 194, HTTPS: 443

\subsection{User Datagram Protocol: UDP}
Nepovezavna storitev, nudi le best-effort: ni zagotavljanja vrstnega reda, ni nadzora zamašitev, ...

Dolžina glave je le 8B.

\begin{tabular}{*{2}{|C{80pt}}|}
	\hline
	\texttt{16b} & \texttt{16b} \\ \hline
	\tiny{\textit{Izvorna vrata}} & \tiny{\textit{Ponorna vrata}} \\ \hline
	\tiny{\textit{Dolžina (skupaj z glavo) v B}} & \tiny{\textit{Internetna kontrolna vsota}} \\ \hline
\end{tabular}

\subsubsection{Internetna kontrolna vsota}
\emph{Pošiljatelj:} pošiljatelj sešteje 16 bitne besede in shrani eniški komplement = kontrolna vsota \\
\emph{Prejemnik:} sešteje 16 bitne besede skupaj s kontrolno vsoto $\to$ dobiti mora same enice

\subsection{Potrjevanje}
\textbf{sprotno potrjevanje}: po vsakem pošiljanju počakaj na potrditev

\textbf{tekoče pošiljanje:} večji razpon za številčenje paketov, shranjevanje paketov na obeh straneh
\begin{itemize}
	\item ponavljanje N potrjenih (\emph{go-back-N})
	\begin{itemize}
		\item pošiljatelj hrani okno največ dovoljenih nepotrjenih paketov
		\item ACK($n$) potrdi vse pakete do vključno $n$
		\item časovna kontrola za najstarejši paket, ko poteče pošlji vse nepotrjene pakete na novo
	\end{itemize}
	\item ponavljanje izbranih (\emph{selective repeat})
	\begin{itemize}
		\item prejemnik shranjuje sporočila in jih sortira pred dostavo
		\item pošiljatel ponovno pošlje le tiste pakete za katere ni prejel potrdila
		\item vsak paket ima svojo časovno kontrolo
	\end{itemize}
\end{itemize}

\textbf{neposredno}: uporaba ACK in NCK
\textbf{posredno}: samo ACK, namesto NCK se ponovi ACK zadnjega segmenta



\subsection{Transfer Control Protocol: TCP}
Povezavna storitev, full duplex, zanesljiv, kontrola pretoka, kontrola zasičenja, \emph{tekoče pošiljanje in neposredno potrjevanje}

\begin{tabular}{*{32}{|C{1pt}}|}
	\hline
	\multicolumn{16}{|c|}{\tiny{\texttt{16b}}} & \multicolumn{16}{c|}{\tiny{\texttt{16b}}} \\ \hline
	\multicolumn{16}{|c|}{\tiny{\textit{Izvorna vrata}}} & \multicolumn{16}{c|}{\tiny{\textit{Ponorna vrata}}} \\ \hline
	\multicolumn{16}{|c|}{\tiny{\textit{Dolžina (skupaj z glavo) v B}}} & \multicolumn{16}{c|}{\tiny{\textit{Internetna kontrolna vsota}}} \\ \hline
	\multicolumn{32}{|c|}{\tiny{\textit{zaporedna št. (B)}}} \\ \hline
	\multicolumn{32}{|c|}{\tiny{\textit{št. potrditve (B)}}} \\ \hline
	%\tiny{\textit{Kontrolna vsota}} & \tiny{\textit{sprejemno okno}} \\ \hline
	\multicolumn{4}{|c|}{\tiny{\textit{Dolzina glave}}} & \multicolumn{3}{c|}{\tiny{\textit{0}}} & \multicolumn{9}{c|}{\tiny{\textit{zastavice}}} & \multicolumn{16}{c|}{\tiny{\textit{sprejemno okno}}} \\ \hline
	\multicolumn{16}{|c|}{\tiny{\textit{Internetna kontrolna vsota}}} & \multicolumn{16}{c|}{\tiny{\textit{karalec urg. data}}} \\ \hline
	\multicolumn{32}{|c|}{\tiny{\textit{opcije (spremenljiva dolžina)}}} \\ \hline
\end{tabular}

Dolžina glave (4b) je podana v številu 32-bitnih besede

Sprejemno okno (16b) št. bajtov, ki jih sprejemnik lahko sprejme

Zastavice (9b):
\begin{itemize}
	\item URG*: urgentni podatki
	\item ACK: potrditev povezave
	\item PSH: takoj predaj aplikaciji
	\item RST, SYN, FIN: vzpostavljanje in rušenje povezave
\end{itemize}

\subsubsection{Vzpostavitev povezave - trojno rokovanje}
\begin{enumerate}
	\item Odjemalec pošlje segment z zastavico \texttt{SYN} (sporoči začetno številko segmenta, ni podatkov)
	\item Strežnik vrne segment \texttt{SYN ACK} (rezervira medpomnilnik, odgovori z začetno številko svojega segmenta)
	\item Odjemalec vrne \texttt{ACK}, lahko že s podatki ("štuporama")
\end{enumerate}

\subsubsection{Rušenje povezave}
\begin{enumerate}
	\item odjemalec pošlje segment  \texttt{TCP FIN} strežniku
	\item strežnik potrdi z  \texttt{ACK}, zapre povezavo, pošlje  \texttt{FIN}
	\item odjemalec prejme strežnikov  \texttt{FIN}, potrdi ga z  \texttt{ACK} (počaka časovni interval, da po potrebi ponovno pošlje  \texttt{ACK}, če se ta izgubi)
	\item strežnik sprejme  \texttt{ACK}, končano
\end{enumerate}

\subsubsection{Nastavitev časovne kontrole}
Na potrditev moramo čakati vsaj \texttt{RTT} (Round Trip Time).

\begin{gather*}
	\texttt{OcenRTT}[i] = (1-\alpha) \texttt{OcenRTT}[i-1] + \alpha  \texttt{IzmerRTT}[i] \\
	\texttt{DevRTT}[i] = (1-\beta) \texttt{DevRTT}[i-1] + \beta  \big| \texttt{IzmerRTT}[i] - \texttt{OcenRTT}[i] \big| \\
	\texttt{ČakalniInterval}[i] = \texttt{OcenRTT}[i] + 4  \texttt{DevRTT}[i]
\end{gather*}

\subsubsection{Način potrjevanja}
TCP uporablja podobno strategijo kot \emph{ponavljanje N nepotrjenih}. Pošiljatelj ima časovno kontrolo le za najstarejši nepotrjen segment, vendar ob poteku časovne kontrole ne pošlje vseh segmentov v oknu, temveč le najstarejši nepotrjen segment.

RFC2018 vpeljuje potrjevanje le izbranih paketov.

\begin{tabularx}{\columnwidth}{|X|X|}
	\hline
	\emph{Dogodek pri prejemniku} & \emph{Odziv prejemnika} \\ \hline
	Sprejem segmenta s pričakovano številko, vsi prejšnji že potrjeni. &
	Počakaj na naslednji segment max 500 ms. Če ta pride v tem intervalu, izvedi zakasnjeno potrditev obeh (\emph{delayed ACK}). Če ne pride v tem intervalu, potrdi samo prejetega. \\ \hline
	Isto kot zgoraj, a potrditev za prejšnji segment še ni bila poslana. &
	Takoj pošlji \emph{kumulativno potrditev} za oba segmenta brez izvajanja zakasnjene potrditve. \\ \hline
	Sprejem segmenta s previsoko številko (\emph{zaznamo vrzel}) &
	Takoj potrdi zadnji še sprejeti segment (\emph{pošlji duplikat ACK}). \\ \hline
	Sprejem segmenta z najnižjo	številko iz vrzeli (\emph{polnjenje vrzeli}) &
	Takoj potrdi segment. \\ \hline
\end{tabularx}

\subsubsection{Hitro ponovno pošiljanje (fast retransmit)}
Ponovno pošiljanje se običajno izvede po preteku časovne kontrole. Če pa pošiljatel prejme 3 potrditve za že potrjen paket, takoj izvede ponovno pošiljanje.

\subsubsection{Kontrola pretoka}
Usklajevanje hitrosti pošiljanja med pošiljateljem in prejemnikom.

Prejemnik sporoča razpolžljiv prostor v pomnilniku v polju 'sprejemno okno':
\begin{align*}
	\texttt{rwnd} = \texttt{velikost} - [\texttt{lastByteRcvd} - \texttt{lastByteRead}]
\end{align*}

\subsubsection{Oznake}
\begin{align*}
	\texttt{rwnd} \quad  \dots & \quad \text{recieve window} \\
	\texttt{cwnd} \quad  \dots & \quad \text{congestion window} \\
	\texttt{MSS} \quad  \dots & \quad \text{maximum segment size} \\
	\texttt{RTT} \quad  \dots & \quad \text{round trip time} \\
	\texttt{ssthresh} \quad  \dots & \quad \text{slow start threshold} \\
\end{align*}

\subsubsection{Nadzor zasičenja}
$\min(\texttt{rwnd}, \texttt{cwnd})$ določa največ nepotrjenih B preden moramo ustaviti pošiljanje.

\textbf{TCP Tahoe:} osnovna verzija; 2 fazi: \emph{počasen začetek} in \emph{izogibanje zasičenju}; po izgubi paketa \texttt{cwnd} = 1 \texttt{MSS}.

\textbf{TCP Reno:} 3 faze: \emph{počasen začetek}, \emph{izogibanje zasičenju} in \emph{hitra obnova}; če dobiš 3 kopije ACK že potrjenih podatkov, \texttt{cwnd} = \texttt{cwnd}/2 + 3 \texttt{MSS}.

\textbf{TCP Vegas:} dodano zaznavanje sitacij, ki vodijo v zasičenje in \emph{linearno zmanjševanje hitrosti} ob zasičenju.


Na začetku: \texttt{cwnd} = 1\texttt{MSS}; faza počasnega začetka;\\ \texttt{ssthresh} = $\infty$.


\begin{itemize}
	\item \textbf{Faza počasnega začetka:}\\
	Za vsak prejet ACK: \texttt{cwnd} += $\min(\texttt{N},\texttt{MSS})$, kjer je \texttt{N} št. B, ki jih je potrdil ta ACK. \\
	\emph{Alt. rešitev}: Za vsak prejet ACK:  \texttt{cwnd} += \texttt{MSS}.

	\emph{Na ta način se skozi čas \texttt{cwnd} povečuje eksponentno.}
	\begin{itemize}
		\item Če je \texttt{cwnd} $\geq$ \texttt{ssthresh}, preidemo v fazo izogibanja zasičenju.
		\item Če prejmemo 3 ACK, že potrjenega segmenta, gremo v \emph{fazo hitre obnove} \texttt{cwnd} = \text{ssthresh} + 3\texttt{MSS}, \texttt{ssthresh} = \texttt{cwnd} / 2
	\end{itemize}
	
	\item \textbf{Faza izogibanja zasičenju:}\\
	Števemo B, ki jih potrdijo ACK. Ko to število preseže \texttt{cwnd}, resetiramo števec in \texttt{cwnd} += \text{MSS}. \\
	\emph{Alt. rešitev}: Za vsak ACK, ki potrdi nove podatke povečamo $\texttt{cwnd} += \texttt{MSS}\cdot\texttt{MSS} / \texttt{cwnd}$
	
	\emph{Na ta način se skozi čas \texttt{cwnd} povečuje linearno.}
	\begin{itemize}
		\item Če poteče časovna pontrola za najstarejši segment, gremo v \emph{fazo počasnega začetka} in  \texttt{cwnd} = \text{MSS}, \texttt{ssthresh} = \texttt{cwnd} / 2.
		\item Če prejmemo 3 ACK, že potrjenega segmenta, gremo v \emph{fazo hitre obnove} \texttt{cwnd} = \text{ssthresh} + 3\texttt{MSS}, \texttt{ssthresh} = \texttt{cwnd} / 2
	\end{itemize}
	
	\item \textbf{Faza hitre obnove:}\\
	Za vsak podvojeni ACK: \texttt{cwnd} += \texttt{MSS}.
	\begin{itemize}
		\item Če dobimo ACK, ki potrjuje nove podatke, gremo v \emph{fazo izogibanja zasičenju} in \texttt{cwnd} += \texttt{ssthresh}.
		\item Če preteče nek timeout, gremo v \emph{fazo počasnega začetka} in  \texttt{cwnd} = \text{MSS}, \texttt{ssthresh} = \texttt{cwnd} / 2.
	\end{itemize}
\end{itemize}


TCP konvergira k pravični delitvi pasovne širine med uporabniki.

\section{Aplikacijska plast}
\subsection{HTTP}

\subsubsection{Glava}
\texttt{Metoda URL Verzija}\\
\texttt{Ime polja 1: vrednost 1}\\
\texttt{...}\\
\texttt{Ime polja n: vrednost n}\\
\textit{[prazna vrstica]}\\
\texttt{TELO}

\subsubsection{Metode}
\begin{tabularx}{\columnwidth}{X X}
	\hline
	\texttt{GET} & zahteva objekta \\ \hline
	\texttt{POST} & zahteva objekta + poslane vrednosti (obrazci) \\ \hline
	\texttt{HEAD} & zahteva, na katero strežnik odgovori z odgovorom brez zahtevanega objekta (uporabno za razhroščevanje) \\ \hline
	\texttt{PUT} & (HTTP 1.1) – nalaganje na strežnik (upload) \\ \hline
	\texttt{DELETE} & (HTTP 1.1) – brisanje s strežnika \\ \hline
	\texttt{TRACE} & razhroščevanje (echo-odmev zahtevka, podobno PING) \\ \hline
	\texttt{CONNECT} & povezava preko medstrežnika \\ \hline
	\texttt{OPTIONS} & povpraševanje o možnih opcijah pri zahtevku \\ \hline
\end{tabularx}

\subsubsection{Statusi}
\begin{itemize}
	\item \textbf{1xx: informativne kode}
	\item \textbf{2xx: uspešno} \\
		200: OK
	\item \textbf{3xx: preusmeritev} \\
		301: Moved Permanently – prestavljen dokument 
	\item \textbf{4xx: napake pri odjemalcu} \\
		400: Bad Request – sintaksa \\
		404: Not Found – ni dokumenta
	\item \textbf{5xx: napake na strežniku} \\
		500: Internal Server Error \\
		505: HTTP Version Not Supported
\end{itemize}

\subsubsection{Vrste HTTP povezav}
\begin{enumerate}
	\item \textbf{nevztrajne (nonpersistent)}: za vsak prenašani objekt (stran, sliko) se vzpostavi nova TCP	povezava
	
	2 RTT/objekt (rokovanje + prenos)
	\item \textbf{vztrajne (persistent)}: strežnik uporabi isto povezavo za pošiljanje več objektov, strežnik pusti povezavo odprto
	
	1 RTT/objekt (prenos)
	\item \textbf{vztrajne s cevovodom (persistent, pipelined)}: tekoče pošiljanje več zahtev naenkrat, brezčakanja na prejem prejšnjih
\end{enumerate}

\subsubsection{Medstrežnik}
Pogojni zahtevek: \texttt{If-modified-since: <datum>}.

Strežnik pošlje nov stran ali pa \texttt{HTTP/1.1 304 Not Modified}

\subsection{File Transfer Protocol - FTP}

Dve ločeni TCP povezavi:
\begin{itemize}
	\item vrata TCP 21 (kontrolna povezava): ukazi za prenos datotek, uporabniško ime/geslo, menjava map, ...
	\item vrata TCP 20 (prenos podatkov) na zahtevo odjemalca strežnik odpre povezavo, po kateri prenaša podatke	
\end{itemize}


\subsubsection{Aktivni in pasivni način}
Aktivni: odjemalec se iz naključnega porta X poveže na strežnik prek porta 21 (kontrolna povezava), nato se strežnik poveže na odjemalca iz porta 20 na naključen port Y.

Pasivni: odjemalec se iz naključnega porta X poveže na strežnik prek porta 21 (kontrolna povezava). Odlemalec pošlje ukaz PASV. Strežnik odgovori z naključnim portom Y, ki ga je odpru za podatkovno povezavo. Odjemalec se nato iz naključnega porta Z poveže na Y.

\subsubsection{Ukazi}
\texttt{USER <ime>}
\texttt{PASS <geslo>}
\texttt{LIST}
\texttt{RETR <ime datoteke>}
\texttt{STOR <ime datoteke>}

\subsubsection{Odgovori}
\begin{itemize}
	\item \texttt{331} Username OK, password required
	\item \texttt{125} Data connection open, transfer starting
	\item \texttt{452} Error writing file
	\item \texttt{425} Can’t open data connection
\end{itemize}

\subsection{Elektronska pošta}
Sporočilo je sestavljeno iz glave, prazne vrstice in telesa.

Komunikacija med poštnimi strežniki: \textbf{SMTP}

Dostop do pošte:
\begin{itemize}
	\item \textbf{POP}: preprost, ne podpira urejanja pošte na strežniku (le lokalno) zato ni primeren za dostop do istega predala iz več naprav.
	
	Ukazi: \texttt{user <ime>, pass <geslo>, list, retr <id>, dele <id>, quit}
	\item \textbf{IMAP}: omogoča urejanje sporočil po mapah na strežniku, možen prenos le dela sporočila
	\item \textbf{HTTP}: prek brskalnika (recimo Gmail)
\end{itemize}

\subsection{Domain Name System - DNS}

Hierarhična organizacija strežnikov
\begin{itemize}
	\item \textbf{Korenski strežniki}: 13 strežnikov (A-M), vsak je replicirana gruča
	\item \textbf{Top Level Domain - TLD} strežniki: \\
	 generične domene: 7 prvotnih (com, edu, gov, mil, org, net, biz); $\sim$1500 dodatnih (info, blog, ...);
	 $\sim$255 domen za države (si, it, de, tv, am, gl, ...)
	\item \textbf{Avtoritativni strežniki}: organizacija z javnimi računalniki (UL: uni-lj)
\end{itemize}

\textbf{Iterativna poizvedba}:
\begin{itemize}
	\item strežnik vrne bodisi končni odgovor ali pa naziv strežnika za naslednje povpraševanje (primer: lokalni strežnik iterativno povpraša ostale)
\end{itemize}
\textbf{Rekurzivna poizvedba}:
\begin{itemize}
	\item strežnik poišče preslikavo imena in vrne odgovor (primer: naša poizvedba lokalnemu strežniku)
	\item razbremenimo končne kliente komunikacije in povpraševanja
	\item možnost centralnega predpomnenja v lokalnem strežniku!
\end{itemize}

\subsubsection{DNS record}
\texttt{(Name, Value, Type, TTL)}

\textbf{Time To Live - TLL}: čas veljavnosti zapisa

\textbf{Type}:
\begin{itemize}
	\item \textbf{A/AAAA} \emph{(addres)} \\
	Name: ime računalnika (poddomena), Value: IPv4/IPv6 naslov 
	\item \textbf{NS} \emph{(name server)} \\
	Name: ime domene, Value: ime avtoritativenga DNS strežnika
	\item \textbf{CNAME} \emph{(cannonical name)} \\
	Name: alias ime, Value: pravo (kanonično) ime
	\item \textbf{MX} \emph{(mail exchange)} \\
	Name: alias poštnega strežnika, Value: pravo ime poštnega strežnika
\end{itemize}

\subsection{P2P storitve}
\subsubsection{BitTorrent}
\textbf{strategija}: sosede vpraša po razpoložljivih koščkih datotek in zahteva najprej tiste manjkajoče, ki so
najbolj redki med sosedi \emph{(rarest first!)}

\textbf{pravičnost}: opazujemo hitrost prejemanja od sosedov in jim pošiljamo koščke s sorazmerno visoko
hitrostjo

pomembna je \textbf{vzpodbuda za sodelovanje}

\subsubsection{Skype}

NAT povzroča težave v P2P arhitekturah, ker zunanji
odjemalci ne morejo direktno kontaktirati odjemalca
za prehodom NAT.

rešitev:
\begin{itemize}
	\item odjemalce A vzpostavi z B zvezo preko nadzornih vozlišč NA in NB,
	\item NA in NB izbereta tretje premostitveno (relay) nadzorno vozlišče (NR), s katerim A in B vzpostavita sejo
	\item premostitveno vozlišče poskuša zagotoviti neposredno povezavo med odjemalcema
\end{itemize}

\section{Sejna plast}
Naloge: 
\begin{itemize}
	\item vzpostavljanje, rušenje, vzdrževanje sej med aplikacijama (potek dialoga med aplikacijama)
	\item odgovornost za obnovitvene točke (checkpoints) seje in obnovo (recovery), če seja ne uspe
	\item sinhronizacija podatkov iz različnih tokov in virov
\end{itemize}
V eni seji je lahko več transportnih povezav in ena transportna povezava lahko sega čez več sej.

\section{Predstavitvena plast}
Storitve
\begin{itemize}
	\item \textbf{predstavitev podatkov} \emph{(binarni tipi)}: različni sistemi predstavljajo binarne tipe na različen način; uporaba sintakse \textbf{ASN.1} (Abstract Syntax Notation 1) zmanjša število vseh možnih preslikav
	\item \textbf{predstavitev alfanumeričnih znakov} \emph{(združljivost kodnih strani)}: znaki so predstavljeni s številkami po kodnem sistemu, potrebno zagotoviti, da se prenašajo pravi znaki
	\item \textbf{stiskanje podatkov}: omogoča zmanjšanje velikosti podatkov in s tem pohitritev prenosa (jpeg, mpeg)
	\item \textbf{zaščita podatkov} \emph{(kriptiranje)}: varnostni mehanizem, ki omogoča vpeljavo zaupnosti v komunikacijo
\end{itemize}

\subsection{Kriptogorafija}
\begin{align*}
	m \quad \dots& \quad \text{čistopis} \\
	K_A(m) \quad \dots& \quad \text{kriptogram, kriptiran s ključem A} \\
	K_B(K_A(m)) \quad \dots& \quad \text{odkriptiran z dekripcijskim ključem B}
\end{align*}


\subsubsection{Metode kriptiranja}
Glede na način kriptiranja (\emph{algoritem}):
\begin{itemize}
	\item \textbf{substitucija} (\emph{zamenjava znakov z drugimi})
	\begin{itemize}
		\item \textbf{Cezarjev kriptogram}: vsako črko zamenjamo z naslednjo $k$-to po abecedi.
		\item \textbf{Vigenèr-jev kriptogram}: Ključ $K$ je neka beseda dolžine $n$. Vsako črko sporočila $M$ zamenjamo:  
		$M[i] = M[i] + K[i \% n]$

		\emph{Računamo v grupi seštevanja ostankov črk.}
		\item \textbf{Porterjev kriptogram}: kriptiramo po 2 zanak skupaj (z veliko abecedo)
	\end{itemize}
	\item \textbf{transpozicija} (\emph{zamenjava vrstnega reda zankov})
	\item \textbf{kombinirane metode}
\end{itemize}

Glede na velikost sporočila:
\begin{itemize}
	\item \textbf{znakovna}
	\item \textbf{bločna}
\end{itemize}

Glede na ključe:

\textbf{Simetrično} enkripcija: $K_S(K_S(m)) = m$.

\textbf{Asimetrična} enkripcija (z javnim in zasebnim ključem): $K_J(K_Z(m)) = m = K_Z(K_J(m))$

\subsection{Bločna kriptografija}
\textbf{Permutacijska škatla} s klujčem $(a b c \dots)$ slika $a$. bit na $0$. mesto, $b$. bit na $1$. mesto, itd.

\textbf{Substitucijska škatla}: dekoder (slika števila v prižgan \emph{en} bit) $\rightarrow$ s-škatla $\rightarrow$ koder (slika \emph{en} prižgan bit v št.)

\subsubsection{Data Encryption Standard - DES}
64b blok $\rightarrow$ transpozicija $T_1$ $\rightarrow$ iteracija 1 $\rightarrow \dots \rightarrow$ iteracija 16 $\rightarrow$ zamenjamo prvo in zadnjo polovico bitov $\rightarrow$ transpozicija $T_1^{-1}$ $\rightarrow$ 64b kriptogram

izhod iteracije $i-1$ razdelimo na polovici $L_{i-1}$ in $R_{i-1}$ $\rightarrow$ $L_i = R_{i-1}$ in $R_i = L_{i-1} \otimes f(R_{i-1}, K_i)$

\subsubsection{Trojni DES}
Sporočilo se kriptira ($K_1$), dekriptira ($K_2$), kriptira ($K_3$).

Združljiv z DES, če $K_2 = K_3$.

\subsubsection{Advanced Encryption Standard - AES}
Najbolj učinkovita in varna metoda. 

Bloki velikosti 128b, ključi pa 128/192/256b.

\subsubsection{Verižno kriptiranje}
Pošiljatelj s prvom sporočilom pošlje inicializacijski vektor $c(0)$.

Naslednja sporočila kriptira: $c(i) = K_S(m(i) \otimes c(i-1))$

Prejemnik sporočila odkriptira: $m(i) = K_S^{-1}(c(i)) \otimes c(i-1)$


\subsection{Asimetrična kriptografija}
RSA:
\begin{itemize}
	\item izberemo veliki praštevili $p, q$, izračunamo:
	\[n = pq \qquad z = (p-1)(q-1)\]
	\item izberemo $e$ tako, da $\gcd(e,z) = 1$
	\item izberemo $d$ tako, da $ed \equiv_z 1$
	\item javni klujč: $(n, e)$, zasebni ključ $(n, d)$
\end{itemize}
\begin{align*}
	\text{kriptiranje}:& m \mapsto c = m^e \text{mod} n \\
	\text{dekriptiranje}:& c \mapsto m = c^d \text{mod} n \\
\end{align*}

\end{multicols}
\end{document}